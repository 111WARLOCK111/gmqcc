\documentclass{article}

%%% PACKAGES
\usepackage{geometry}
\usepackage[utf8]{inputenc}
\usepackage[parfill]{parskip}
\usepackage{subfig}
\usepackage{listings}
\usepackage{color}
\usepackage{sectsty}

%%% GEOMETRY FOR DOCUMENT
\geometry{a4paper}

%%% HEADERS/FOOTERS APPEARANCE
\usepackage{fancyhdr} % This should be set AFTER setting up the page geometry
\pagestyle{fancy} % options: empty , plain , fancy
\renewcommand{\headrulewidth}{0pt} % customise the layout...
\lhead{}\chead{}\rhead{}
\lfoot{}\cfoot{\thepage}\rfoot{}

%%% SECTION TITLE APPEARANCE
\allsectionsfont{\sffamily\mdseries\upshape} % (See the fntguide.pdf for font help)

%%% ToC APPEARANCE
\usepackage[nottoc,notlof,notlot]{tocbibind} % Put the bibliography in the ToC
\usepackage[titles,subfigure]{tocloft} % Alter the style of the Table of Contents
\renewcommand{\cftsecfont}{\rmfamily\mdseries\upshape}
\renewcommand{\cftsecpagefont}{\rmfamily\mdseries\upshape} % No bold!

%%% listing language definitions
%%% BNF for now, QuakeC will be later
\definecolor{keyword1}{RGB}{0,102,153}
\definecolor{keyword2}{RGB}{0,153,102}
\definecolor{keyword3}{RGB}{0,153,255}
\definecolor{comment}{RGB}{204,0,0}
\definecolor{function}{RGB}{153,102,255}
\definecolor{digit}{RGB}{255,0,0}
\definecolor{string}{RGB}{255,0,204}
\definecolor{rule}{RGB}{192,192,192}
\definecolor{back}{RGB}{250,250,250}

\lstdefinelanguage{bnf}{
  keywordstyle={\color{keyword2}\bfseries},
  keywords={},
  otherkeywords={::=,|},
  morecomment=[s][\color{comment}]{(*}{*)},
  stringstyle=\color{string},
  showstringspaces=false,
  frame=none,
  rulecolor=\color{rule},
  backgroundcolor=\color{back}
}

%% Title Information %%
\title{The GMQCC QuakeC Programming Language}
\author{Dale Weiler}
\date{\today}

\begin{document}

%% Title Page %%
\maketitle
\thispagestyle{empty}
\raggedright
\abstract
This document specifies the form and establishes the interpretation of programs written in
the GMQCC QuakeC programming language variant (refereed simply as QuakeC throughout this
document). It specifies:
\begin{itemize}
	\item the representation of QuakeC programs;
	\item the syntax and constraints of the QuakeC language;
	\item the semantic rules for interpreting QuakeC programs;
	\item the representation of input data to be processed by QuakeC programs;
	\item the representation of output data produced by QuakeC programs;
	\item the restrictions and limits imposed by a conforming implementation of QuakeC.
\end{itemize}
This document does not specify
\begin{itemize}
	\item the mechanism by which QuakeC programs are transformed for use by a data-
	      processing system;
	\item the mechanism by which QuakeC programs are invoked for use by a data-processing
	      system;
	\item the mechanism by which input data are transformed for use by a QuakeC program;
	\item the size or complexity of a program and its data that will exceed the capacity
	      of any specific data-processing system or the capacity of a particular
	      execution environment;
	\item all minimal requirements of a data-processing system that is capable of
	      supporting a conforming implementation.
\end{itemize}

%% Table Of Contents %%
\newpage
\thispagestyle{empty}
\tableofcontents
\newpage

%% Begin Contents %%
\raggedright % No weird TEX spacing on lines to fill page

%% -> Terms, definitions, and symbols %%
\section{Terms, definitions, and symbols}
\subsection*{argument}
Expression in the comma-separated list bounded by the parentheses in a function call
expression, or a sequence of preprocessing tokens in the comma-separated list bounded
by the parentheses in a function-like macro invocation.

\subsection*{behavior}
External appearance or action

\subsection*{implementation-defined behavior}
Unspecified behavior where each implementation documents how the choice is made.

\subsection*{undefined behavior}
Behavior, upon use of a non-portable or erroneous program construct or of erroneous data,
for which this document imposes no actual requirements.

\subsection*{unspecified behavior}
Use of an unspecified value, or other behavior where this document provides two or more
possibilities and imposes no further requirements on which is chosen in any instance.

\subsection*{constraint}
Restriction, either syntactic or semantic, by which the exposition of language elements
is to be interpreted.

\subsection*{diagnostic message}
Message belonging to an implementation-defined subset of the implementation's message
output.

\subsection*{object}
Region of data storage in the execution environment, the contents of which can represent
values.

\subsection*{parameter}
Object declared as part of a function declaration or definition that acquires a value on
entry to the function, or an identifier from the comma-separated list bounded by the
parentheses immediately following the macro name in a function-like macro definition.

\subsection*{recommended practice}
Specification that is strongly recommended as being in keeping with the intent of this
document, but that may be impractical for some implementations.

\subsection*{value}
Precise meaning of the contents of an object when interpreted as having a specific type.

\subsection*{implementation}
Particular set of software, running in a particular translation environment under
particular control options, that performs translation of programs for, and supports
execution of functions in, a particular execution environment.

\subsection*{implementation-defined value}
Unspecified value where each implementation documents how the choice is made.

\subsection*{unspecified value}
Valid value of the relevant type where this document imposes no requirements on which
value is chosen in any instance.

%% -> Conformance %%
\section{Conformance}
In this document, "shall" is to be interpreted as a requirement on an implementation
or on a program; conversely, "shall not" is to be interpreted as a prohibition. \\
If a "shall" or "shall not" requirement that appears outside of a constraint is violated,
the behavior is undefined.  Undefined behavior is otherwise indicated in this document by
the words "undefined behavior" or by the omission of any explicit definition of behavior.
There is no difference in emphasis among these three; they all describe "behavior that is
undefined".

%% -> Enviroment %%
\section{Environment}
An implementation that translates QuakeC source files and executes QuakeC programs in two
data processing-system environments, which will be called the translation environment and
the execution environment in this document. Their characteristics define and constrain the
results of executing QuakeC programs constructed according to the syntactic and semantic
rules for conforming implementations.
\subsection{Conceptual models}
\subsubsection{Translation environment}
\paragraph*{Translation steps}
The precedence among the syntax rules of translation is specified by the following steps
\begin{enumerate}
	\item Physical source file characters are mapped, in an implementation-defined manner,
	      to the source character set (introducing new-line characters for end-of-line
	      indicators) if necessary.  Trigraph and digraph sequences are replaced by their
	      corresponding single-character internal representations.
	\item The source file is decomposed into preprocessing tokens and sequences of white-
	      space characters (including comments). A source file shall not end in a partial
	      preprocessing token or in a partial comment. Each comment is replaced by one
	      space character. New-line characters are retained. Whether each nonempty
	      sequences of white-space characters other than new-line is retained or replaced
	      by one space character is implementation-defined.
	\item Preprocessing directives are executed, macro invocations are expanded 
	      recursively. A \#include preprocessing directive causes the named header or
	      source file to be processed from step one through step three, recursively. All
	      preprocessing directives are then deleted.
	\item Each source character set member and escape sequence in character constants and
	      string literals is converted to the corresponding member of the execution
	      character set; if there is no corresponding member, it is converted to an
	      implementation-defined member other than the null character.
	\item Adjacent string literal tokens are concatenated.
	\item White-space characters seperating tokens are no longer significant. Each
	      preprocessing token is converted into a token. The resulting tokens are then
	      syntactically and semantically analyzed and translated.	    		      
\end{enumerate}
\subparagraph*{Footnotes}
Implementations shall behave as if these steps occur separately, even though many are likely
to be folded together in practice. Source files need not be stored as file, nor need there
be any one-to-one correspondence between these items and any external representation. The
description is conceptual only, and does not specify any particular implementation.

\paragraph*{Diagnostics}
A conforming implementation shall produce at least on diagnostic message(identified in an
implementation-defined manner) if a source file contains a violation of any syntax rule or
constraint, even if the behavior is also explicitly specified as undefined or
implementation-defined. Diagnostic messages need not be produced in other circumstances.

%% ->-> Execution environments %%
\subsubsection{Execution environment}
A conforming execution environment shall provide at minimal the following 15 definitions
for built in functions, with an accompanying header or source file that defines them.
\begin{enumerate}
	\item entity ()                                spawn
	\item void   (entity)                          remove
	
	\item string (float)                           ftos
	\item string (vector)                          vtos
	\item string (entity)                          etos
	\item float  (string)                          stof
	
	\item void   (string, ...)                     dprint
	\item void   (entity)                          eprint
	
	\item float  (float)                           rint
	\item float  (float)                           floor
	\item float  (float)                           ceil
	\item float  (float)                           fabs
	\item float  (float)                           sin
	\item float  (float)                           cos
	\item float  (float)                           sqrt
\end{enumerate}
The numbers of which these built-ins are assigned is implementation-defined;
an implementation is allowed to use these built-ins however it sees fit.

\pagebreak
%% -> Language %%
\section{Language}
\subsection{Notation}
The syntax notation used in this document is that of a BNF specification.  A set of
derivation rules, often written as:
\begin{lstlisting}[language=bnf]
	symbol ::= expression
\end{lstlisting}
Where symbol is a nonterminal, and the expression consists of one or more sequences of
symbols; more sequences are separated by a vertical bar \textbar, indicating a choice,
the whole being a possible substitution for the symbol on the left.  Symbols that never
appear on the left side are terminals.
\linebreak

This document defines language syntax throughout it's way at defining language
constructs If you're interested in a summary of the language syntax, one is given in 
annex A.

%% -> Concepts %%
\subsection{Concepts}
%% ->-> Scopes of identifiers %%
\subsubsection{Scopes of identifiers}
An identifier can denote an object; a function, or enumeration; a label name; a macro
name; or a macro parameter. The same identifier can denote different items at different
points in the program. A member of an enumeration is called an enumeration constant.
Macro names and macro parameters are not considered further here, because prior to the
semantic phase of program translation any occurrences of macro names in the source file
are replaced by the preprocessing token sequences that constitute their macro definitions.
\linebreak

For each different item that an identifier designates, the identifier is visible (i.e, 
can be used) only within a region of program text called its scope.  Different items 
designated by the same identifier either have different scopes, or are in different name
spaces. There are four kinds of scopes: function, file, block and function prototype.
(A function prototype is a declaration of a function that declares the types of its
parameters.)
\linebreak

A label name is the only kind of identifier that has function scope.  It can be used (in 
a goto statement) anywhere in the function in which it appears, and is declared
implicitly by its syntactic appearance (prefixed by a colon :, and suffixed with a 
statement).
\linebreak

Every other identifier has scope determined by the placement of its declaration (in a 
declarator or type specifier).  If the declarator or type specifier that declares the 
identifier appears outside any block or list of parameters, the identifier has file 
scope,  which terminates at the end of the file.  If the declartor or type specifier that
declares the identifier appears inside a block or within the list of parameter
declarations in a  function definition, the identifier has block scope, which terminates
at the end of the  associated block.  If the declarator or type specifier that declares 
the identifier appears  within the list of parameter declarations in a function prototype
(not part of a function definition), the identifier has function prototype scope, which 
terminates at the end of the function declarator.  If an identifier designates two 
different items in the same  name space, the scopes might overlap.  If so, the scope of 
one item (the inner scope) will be a strict subset of the scope of the other item (the 
outer scope). Within the inner scope, the identifier designates the item declared in the
inner scope; the item declared in the  outer scope is hidden (and not visible) within 
the inner scope.
\linebreak

Unless explicitly stated otherwise, where this document uses the term "identifier" to 
refer to some item (as opposed to the syntactic construct), it refers to the item in the 
relevant name space whose declaration is visible at the point the identifier occurs.
\linebreak

Two identifiers have the same scope if and only if their scopes terminate at the same 
point.
\linebreak

Each enumeration constant has scope that begins just after the appearance of its defining
enumerator in an enumerator list. Any other identifier has scope that begins just after 
the completion of its declarator.

%% ->-> Name spaces of identifiers %%
\subsubsection{Name spaces of identifiers}
If more than one declaration of a particular identifier is visible at any point in a 
source file, the syntactic context disambiguates uses that refer to different items. 
Thus, there  are separate name spaces for various categories of identifiers, as follows:
\linebreak
\begin{itemize}
	\item Label names (disambiguated by the syntax of the label declaration and use);
	\item Enumerations (disambiguated by following the keyword enum);
	\item All other identifiers, called ordinary identifiers (declared in ordinary
	      declarators or as enumeration constants).
\end{itemize}

%% ->-> Types %%
\subsubsection{Types}
The meaning of a value stored in an object returned by a function is determined by the 
type  of the expression used to access it. (An identifier declared to be an object is the simplest
such expression; the type is specified in the declaration of the identifier.)  Types are 
partitioned into object types (types that fully describe objects), function types(types 
that  describe functions), and incomplete types(types that describe objects but lack 
information).
\linebreak

An object declared type bool is large enough to store the values 0 and 1.
\linebreak

An object declared type float is a real type; An object declared type vector is a 
comprised set of three floats that respectively represent the \underline{x,y,z} 
components of a three-dimensional vector.
\linebreak

An enumeration comprises a set of named integer constant values.  Each distinct 
enumeration constitutes a different enumerated type.
\linebreak

Enumeration types and float are collectively called arithmetic types. Each arithmetic 
type belongs to one type domain.
\linebreak

The void type comprises an empty set of values; it is an incomplete type that cannot be
completed.
\linebreak

A number of derived types can be constructed from the object, function and incomplete 
types, as follows:
\linebreak

\begin{itemize}
	\item An array type describes a contiguously allocated nonempty set of objects with a 
          particular object type, called the element type. Array types are characterized 
          by their element type and by the number of elements in the array.  An array type
          is said to be derived	from its element type, and if its element is type T, the 
          array type is sometimes called "array of T".  The construction of an array type
          from an element type is called "array type derivation".
	\item A function type describes a function with a specified return type. A function
          type is characterized by its return type and the number and types of its
          parameters. A function type is said to be derived from its return type, and if 
          its return type is T, the function type is sometimes called "function returning
          T". The construction of a function type from a return type is called "function
          type derivation".
\end{itemize}

Arithmetic types are collectively called scalar types. Arrays and vectors are 
collectively called aggregate types.
\linebreak

An array of unknown size is an incomplete type.  It is completed, for an identifier of 
that type, by specifying the size in a later declaration.  Arrays are required to have 
known constant size.
\linebreak

A type is characterized by its type category, which is either the outermost derivation 
of a derived type (as noted above in the construction of derived types), or the type 
itself if the type consists of no derived types.
\linebreak

Any type so far mentioned is an unqualified type.  Each unqualified type has several 
qualified versions of its type, corresponding to the combinations of one, two, or all 
two of const and volatile qualifiers.  The qualified or unqualified versions of a type 
are distinct types that belong to the same type category and have the same representation.
A derived type is not qualified by the qualifiers (if any) of the type from which it 
is derived.
\linebreak

%% ->-> Compatible types and composite type %%
\subsubsection{Compatible types and composite type}
Two types have compatible type if their types are the same.
\linebreak

All declarations that refer to the same object or function shall have compatible type; 
otherwise the behavior is undefined.
\linebreak

A composite type can be constructed from two types that are compatible; it is a type that 
is compatible with both of the two types and satisfies the following conditions:
\begin{itemize}
	\item If one type is an array, the composite type is an array of that size.
	\item If only one type is a function type with a parameter type list(a function 
	      prototype), the composite type is a function prototype with the parameter type 
	      list.
	\item If both types are function types with parameter type lists, the type of each 
	      parameter in the composite parameter type list is the composite type of the 
	      corresponding parameters.
\end{itemize}
These rules apply recursively to types from which the two types are derived.
\linebreak

%% ->Conversions %%
\subsection{Conversions}
Several operators convert operand values from one type to another automatically. This 
sub-clause specifies the result required from such an implicit conversion.
\linebreak

Conversion from an operand value to a compatible type causes no change to the value or
the representation.
\linebreak

TODO: Specify all implicit conversions.

%% ->->Aritmetic operands %%
\subsubsection{Arithmetic operands}
\paragraph*{Boolean type}
When any scalar value is converted to bool, the result is 0 if the value compares equal 
to 0; otherwise the result is 1.

%% ->->Other operands %%
\subsubsection{Other operands}
\paragraph{Lvalues, arrays and function designators}
An lvalue is an expression with an object type or an incomplete type other than void; 
if an lvalue does not designate an object when it is evaluated, the behavior is undefined.
When an object is said to have a particular type, the type is specified by the lvalue 
used to designate the object.  A modifiable lvalue is an lvalue that does not have an 
array type, does not have an incomplete type, and does not have a const-qualified type.
\linebreak

Except when it is the operand of the unary \& operator, the ++ operator, the -- operator,
or the left operand of the . operator or an assignment operator, an lvalue that does not 
have array type is converted to the value stored in the designated object (and is no 
longer an lvalue).  If the lvalue has qualified type, the value has the unqualified 
version of the type of the lvalue; otherwise, the value has the type of the lvalue. If 
the lvalue has an incomplete type and does not have array type, the behavior is undefined.
\linebreak

A function designator is an expression that has function type.

\paragraph*{void}
The (nonexistent) value of a void expression (an expression that has type void) shall not 
be used in any way, and implicit conversions (except to void) shall not be applied to 
such an expression.  If an expression of any other type is evaluated as a void expression,
its value or designator is discarded. (A void expression is only evaluated for its 
side effects.)
\pagebreak

\subsection{Lexical elements}
\paragraph*{Syntax}
\begin{lstlisting}[language=bnf]
token ::= keyword    
        | identifier
        | constant   
        | string-literal
        | punctuator
preprocessing-token ::= header-name 
                      | identifier
                      | pp-number   
                      | string-literal
                      | punctuator
\end{lstlisting}
\paragraph*{Constraints}
Each preprocessing token that is converted to a token shall have the lexical form of a 
keyword, an identifier, a constant, a string literal, or a punctuator.

\paragraph*{Semantics}
A token is the minimal lexical element of the language in translation steps six and seven.
The categories of tokens are: keywords, identifiers, constants, string literals, and
punctuators. A preprocessing token is the minimal lexical element of the language in
translation steps three through five. The categories of preprocessing tokens are: header
names, identifiers, preprocessing numbers, string literals, punctuators and other single
non-white-space characters that do not lexically match the other preprocessing token
categories. If a ' or a " character matches the last category, the behavior is undefined.
Preprocessing tokens can be separated by white space; this consists of comments (described
later), or white-space characters (space, horizontal tab, new-line, vertical tab, and form
-feed), or both. In certain circumstances during translation step four, white space (or
the absence thereof) serves as more than preprocessing token separation. White space may 
appear within a preprocessing token only as part of a header name or between the quotation
characters in a string literal.
\linebreak

If the input stream has been parsed into preprocessing tokens up to a given character, the
next preprocessing token is the longest sequence of characters that could constitute a
preprocessing token. There is one exception to this rule: header name preprocessing tokens
are recognized only within \#include preprocessing directives and in implementation-defined
locations within \#pragma directives. In such contexts, a sequence of characters that
could be either a header name or string literal is recognized as the former.

%% ->-> Keywords %%
\subsubsection{Keywords}
\paragraph*{Syntax}
\begin{lstlisting}[language=bnf]
keyword ::= enum     | break  | return   | void
          | case     | float  | volatile | for
          | while    | const  | goto     | bool
          | continue | if     | static   | default
          | inline   | do     | switch   | else
          | vector   | entity
\end{lstlisting}
\paragraph*{Semantics}
The above tokens (case sensitive) are reserved (in translation step seven and eight) for 
use as keywords, and shall not be used otherwise.

%% ->->Identifiers %%
\subsubsection{Identifiers}
\begin{lstlisting}[language=bnf]
identifier ::= nondigit
             | identifier nondigit
             | identifier digit
             
nondigit ::= _ | a | b | c | d | e | f | g | h | i 
           | j | k | l | m | n | o | p | q | r | s
           | t | u | v | w | x | y | z | A | B | C
           | D | E | F | G | H | I | J | K | L | M
           | N | P | Q | R | S | T | U | V | W | X
           | Y | Z

digit ::= 0 | 1 | 2 | 3 | 4 | 5 | 6 | 7 | 8 | 9
\end{lstlisting}
\paragraph*{Semantics}
An identifier is a sequence of nondigit characters (including the underscore \_, the lower
case and upper case Latin letters, and other characters) and digits, which designates one
or more items. Lowercase and uppercase letters are distinct. There is a specific limit of
65535 characters for an identifier.
\linebreak

When preprocessing tokens are converted to tokens during translation step six, if a 
preprocessing token could not be converted to either a keyword or an identifier, it is
converted to a keyword.

\paragraph*{Predefined identifiers}
Any identifiers that begin with the prefix \_\_builtin, or are within the reserved name
space are reserved by the implementation.

%% ->->Constants %%
\subsubsection{Constants}
\begin{lstlisting}[language=bnf]
constant             ::= integer-constant
                       | floating-constant
                       | enumeration-constant
                       | character-constant
                       | vector-constant
           
integer-constant     ::= decimal-constant
                       | octal-constant
                       | hexadecimal-constant
                   
decimal-constant     ::= nonzero-digit
                       | decimal-constant digit
                   
octal-constant       ::= 0
                       | octal-constant octal-digit
                   
hexadecimal-constant ::= hexdecimal-prefix
                         hexadecimal-digit
                       | hexadecimal-digit
                         hexadecimal-constant

hexadecimal-prefix:  ::= 0x | 0X

nonzero-digit        ::= 1 | 2 | 3 | 4 | 5 | 6 | 7 | 8
                       | 9
                       
octal-digit          ::= 0 | 1 | 2 | 3 | 4 | 5 | 6 | 7

hexadecimal-digit    ::= 0 | 1 | 2 | 3 | 4 | 5 | 6 | 7
                       | 8 | 9 | a | b | c | d | e | f
                       | A | B | C | D | E | F
\end{lstlisting}

%% ->-> String literals %%
\subsubsection{String literals}
\begin{lstlisting}[language=bnf]
string-literal  := " s-char-sequence "

s-char-sequence := s-char
                 | s-char-sequence s-char
                 
s-char          := ` | ! | @ | # | $ | % | ^ | & | *
                 | ( | ) | _ | - | + | = | { | } | [
                 | ] | | | : | ; | ' | < | , | > | .
                 | ? | / | 1 | 2 | 3 | 4 | 5 | 6 | 7
                 | 8 | 9 | 0 | q | w | e | r | t | y
                 | u | i | o | p | a | s | d | f | g
                 | h | j | k | l | z | x | c | v | b 
                 | n | m | Q | W | E | R | T | Y | U
                 | I | O | P | A | S | D | F | G | |
                 | H | J | K | L | Z | X | C | V | B
                 | N | M
\end{lstlisting}
\paragraph*{Description}
A character string literal is a sequence of zero or more characters enclosed in 
double-quotes, as in "xyz".
\linebreak

The same considerations apply to each element of the sequence in a character string
literal as if it where an integer character constant, except that the single-quote 
' is representable either by itself or by the escape sequence \textbackslash', but 
the double-quote " shall be represented by the escape sequence \textbackslash".

\paragraph*{Semantics}
In translation stage six, the character sequences specified by any sequence of adjacent 
character string literal tokens are concatenated into a single character sequence.

%% ->-> Punctuators %%
\subsubsection{Punctuators}
TODO: BNF

A punctuator is a symbol that has independent syntactic and semantic significance. 
Depending on context, it may specify an operation to be performed (which in turn 
may yield a value or a function designator, produce a side effect, or some combination 
thereof) in which case it is known as an operator (other forms of operator also exist 
in some contexts). An operand is an item on which an operator acts.
\linebreak

TODO: Trigraphs \& Digraphs

\subsubsection{Header names}
TODO
\subsubsection{Preprocessing numbers}
TODO
\subsubsection{Comments}
Except within a character constant, a string literal, or a comment, the characters /* 
introduce a comment.  The contents of such a comment are examined only to identify 
characters and to find the characters */ that terminate it.
\linebreak

Except within a character constant, a string literal, or a comment, the characters // 
introduce a comment that includes all characters up to, but not including, the next 
new-line character.  The contents of such a comment are examined only to identify 
characters and to find the terminating new-line character.
\linebreak

%% -> Expressions %%
\subsection{Expressions}
An expression is a sequence of operators and operands that specifies computation of a 
value, or that designates an object or function, or that generates side effects, or that 
performs a combination thereof.
\linebreak

Between the previous and next sequence point an object shall have its stored value 
modified at most once by the evaluation of an expression. Furthermore, the prior value 
shall be read only to determine the value to be stored.
\linebreak

The grouping of operators and operands is indicated by the syntax. Except as specified 
later (for the function call (), \&\&, \textbar\textbar ?:, and comma operators), the 
order of evaluation of sub-expressions and the order in which side effects take place 
are both unspecified.
\linebreak

Some operators (the unary \textasciitilde operator, and the binary operators \textless
\textless, \textgreater\textgreater, \&, \^, and \textbar, collectively describe bitwise 
operators) are required to have operands that are either integer, or floating point with
zero points of decimal precision.
\linebreak

If an exceptional condition occurs during the evaluation of an expression (that is, if 
the result is not mathematically defined or not in the range or representable values for 
its type), the behavior is undefined.

%% ->-> Primary expressions %%
\subsubsection{Primary expressions}
\paragraph*{Syntax}
\begin{lstlisting}[language=bnf]
primary-expression ::= identifier
                     | constant
                     | string-literal
                     ( expression )
\end{lstlisting}
\paragraph*{Semantics}
An identifier is a primary expression, provided it has been declared as designating an 
object(in which case it is an lvalue) or a function(in which case it is a function 
designator).
\linebreak

A constant is a primary expression. Its type depends on its form and value.
\linebreak

A string literal is a primary expression.  It is an lvalue.
\linebreak

A parenthesized expression is a primary expression.  Its type and value identical to 
those of the unparenthesized expression.  It is an lvalue, a function designator, or a 
void expression if the unparenthesized expression is, respectively, an lvalue, a 
function designator, or a void expression.

\bibliographystyle{abbrv}
\bibliography{main}

\end{document}
